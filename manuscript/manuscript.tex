\documentclass[]{article}
\usepackage{lmodern}
\usepackage{amssymb,amsmath}
\usepackage{ifxetex,ifluatex}
\usepackage{fixltx2e} % provides \textsubscript
\ifnum 0\ifxetex 1\fi\ifluatex 1\fi=0 % if pdftex
  \usepackage[T1]{fontenc}
  \usepackage[utf8]{inputenc}
\else % if luatex or xelatex
  \ifxetex
    \usepackage{mathspec}
  \else
    \usepackage{fontspec}
  \fi
  \defaultfontfeatures{Ligatures=TeX,Scale=MatchLowercase}
\fi
% use upquote if available, for straight quotes in verbatim environments
\IfFileExists{upquote.sty}{\usepackage{upquote}}{}
% use microtype if available
\IfFileExists{microtype.sty}{%
\usepackage[]{microtype}
\UseMicrotypeSet[protrusion]{basicmath} % disable protrusion for tt fonts
}{}
\PassOptionsToPackage{hyphens}{url} % url is loaded by hyperref
\usepackage[unicode=true]{hyperref}
\hypersetup{
            pdftitle={Body Mass index effects on statin prescribtions},
            pdfborder={0 0 0},
            breaklinks=true}
\urlstyle{same}  % don't use monospace font for urls
\usepackage[margin=1in]{geometry}
\usepackage{longtable,booktabs}
% Fix footnotes in tables (requires footnote package)
\IfFileExists{footnote.sty}{\usepackage{footnote}\makesavenoteenv{long table}}{}
\usepackage{graphicx,grffile}
\makeatletter
\def\maxwidth{\ifdim\Gin@nat@width>\linewidth\linewidth\else\Gin@nat@width\fi}
\def\maxheight{\ifdim\Gin@nat@height>\textheight\textheight\else\Gin@nat@height\fi}
\makeatother
% Scale images if necessary, so that they will not overflow the page
% margins by default, and it is still possible to overwrite the defaults
% using explicit options in \includegraphics[width, height, ...]{}
\setkeys{Gin}{width=\maxwidth,height=\maxheight,keepaspectratio}
\IfFileExists{parskip.sty}{%
\usepackage{parskip}
}{% else
\setlength{\parindent}{0pt}
\setlength{\parskip}{6pt plus 2pt minus 1pt}
}
\setlength{\emergencystretch}{3em}  % prevent overfull lines
\providecommand{\tightlist}{%
  \setlength{\itemsep}{0pt}\setlength{\parskip}{0pt}}
\setcounter{secnumdepth}{5}
% Redefines (sub)paragraphs to behave more like sections
\ifx\paragraph\undefined\else
\let\oldparagraph\paragraph
\renewcommand{\paragraph}[1]{\oldparagraph{#1}\mbox{}}
\fi
\ifx\subparagraph\undefined\else
\let\oldsubparagraph\subparagraph
\renewcommand{\subparagraph}[1]{\oldsubparagraph{#1}\mbox{}}
\fi

% set default figure placement to htbp
\makeatletter
\def\fps@figure{htbp}
\makeatother


\title{Body Mass index effects on statin prescribtions}
\author{}
\date{\vspace{-2.5em}}

\begin{document}
\maketitle

{
\setcounter{tocdepth}{2}
\tableofcontents
}
\section{introduction}

Body Mass Index (BMI) plays an important rule in predicting heart
disease risk(Katzmarzyk et al. 2012). Our goal is to compare 90Th
quantile regression for BMI in Diabetes and prediabetes population.
Quantile regression has many succesfyl application in ecology where
different factores interect in a complicated way that produce different
variation of one factor for different levels of another variabels(Cade
and Noon 2003).

\section{Quantile Regression}

Quantile regression is an important tool used to regress the dependent
variable with high variance over the independent variables. QR is
developed to study the relationships between variables that have week or
no-relationshipes between their means. Quantile regression is more
robust for an outlier than ordinaty least squares regression(OLS).
Moreover, QR is . The independent variables are Gender, Age, race,
diabetes status, and statin use.

For a random variable \(X\), the cumulative distrubution function (CDF)
is \[F(X)=P(X\leq x)\], and the \(\tau\)th quantile of \(X\) is defined
by \[ F^{-1}(\tau)=\text{inf}\{x: F(x)\ge \tau\} \] where \(0<\tau<1\).
Let the loss function is defined by
\[\rho_{\tau}(u)= u(\tau-I_{(u<0)})\] where \(I\) is the indicator
function (Koenker 2005). The quantile estimator is the value that
manimizes the expected loss function

\(E\rho_{\tau}(X-\hat{x})=(\tau-1)\int_{-\infty}^{\hat{x}} (x-\hat{x})dF(x)+\tau\int_{\hat{x}}^{-\infty} (x-\hat{x})dF(x).\)
Differentiating with respect to \(\hat{x}\), we get
\[ 0  =(\tau-1)\int_{-\infty}^{\hat{x}} dF(x)+\tau\int_{\hat{x}}^{-\infty} dF(x)
=F(\hat{x})-\tau.\] Due to monotnicity of the cumulative distribution
function, any solution that satisfies \(\{x:F(x)=\tau \}\) is a
minimzing for the expected loss function.

Least square method expresses conditional mean of y given x as
\(\mu(x)=x^T\beta\) and it solves
\[ \underset{\beta\in \mathcal{R}^p}{\text{min}}\sum_{i=1}^n(y_i- x_i^T\beta)^2.\]
Quantile regression expresses conditional quantile function
\(Q_y(\tau|x)=x^T \beta ({\tau)}\) and solve
\[ \underset{\beta\in \mathcal{R}^p}{\text{min}}\sum_{i=1}^n\rho_{\tau}(y_i- x_i^T\beta)^2.\]This
minimization problem can be reformalated to a linear programming problem
\[ \underset{\beta\in \mathcal{R}^p}{\text{min}} \]

\section{methods}

prevalence and incidence as a function of age (18 -- 84 years in 1-year
intervals), race/ethnicity (non-Hispanic white, non-Hispanic black,
Hispanic, or other), sex, and BMI (underweight, 18.5 kg/m2; normal
weight, 18.5 to 25 kg/m2; overweight, 25 to 30 kg/ m2; obese, 30 to 35
kg/m2; and very From the figure we see

\section{Data}\label{data}

The data used in this study is National Health and Nutrition Examination
Survey data (NHANES)(Disease Control and (CDC) 2018). The data has

\begin{figure}

{\centering \includegraphics[width=0.8\linewidth]{C:/Users/Muhannad/Desktop/multvariat/R_codes/tonton/BMI/images/exploratory} 

}

\caption{A better figure caption}\label{fig:unnamed-chunk-1}
\end{figure}

However,.

\begin{figure}

{\centering \includegraphics[width=0.8\linewidth]{C:/Users/Muhannad/Desktop/multvariat/R_codes/tonton/BMI/images/exploratory2} 

}

\caption{A better figure caption}\label{fig:unnamed-chunk-2}
\end{figure}

So,

\begin{figure}

{\centering \includegraphics[width=0.8\linewidth]{C:/Users/Muhannad/Desktop/multvariat/R_codes/tonton/BMI/images/quant90} 

}

\caption{A better figure caption}\label{fig:unnamed-chunk-3}
\end{figure}

\section*{References}\label{references}
\addcontentsline{toc}{section}{References}

\hypertarget{refs}{}
\hypertarget{ref-cade2003}{}
Cade, Brian S, and Barry R Noon. 2003. ``A Gentle Introduction to
Quantile Regression for Ecologists.'' \emph{Frontiers in Ecology and the
Environment} 1 (8). Wiley Online Library: 412--20.

\hypertarget{ref-NHANES}{}
Disease Control, Centers for, and Prevention (CDC). 2018. ``National
Health and Nutrition Examination Survey Data (Nhanes.''

\hypertarget{ref-katbody}{}
Katzmarzyk, Peter T, Bruce A Reeder, Susan Elliott, Michel R Joffres,
Punam Pahwa, Kim D Raine, Susan A Kirkland, and Gilles Paradis. 2012.
``Body Mass Index and Risk of Cardiovascular Disease, Cancer and
All-Cause Mortality.'' \emph{Canadian Journal of Public Health} 103 (2).
Springer: 147--51.

\hypertarget{ref-koenker2005}{}
Koenker, Roger. 2005. ``Quantile Regression, Volume 38 of.''
\emph{Econometric Society Monographs}.

\end{document}
